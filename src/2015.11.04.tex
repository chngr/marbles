\documentclass{axolotl}
\title{Fundamental Groups and some Properties}
\author{Raymond Cheng}
\date{November 4, 2015}
\begin{document}
\maketitle
\epoint{Fundamental Group of a Circle}
Yesterday, we defined the fundamental group \(\pi_1(X,x)\) of a topological
space \(X\) with respect to the base point \(x \in X\). Its definition is
pretty simple: as a set, it consists of all homotopy classes of curves in \(X\)
starting and ending at \(x\), and the product of two such curves is simply
concatenation of the curves. Despite the simple definition---or perhaps due to
the simple definition---the computation of these groups is ordinarily quite
difficult. Indeed, really the only case in which the fundamental group is simple
to compute is when \(\pi_1(X,x)\) is trivial. The next easiest case is when
\(X = \bS^1\) is the circle, in which case
\[ \pi_1(\bS^1,x) \cong \ZZ, \]
where the isomorphism can be constructed by sending the homotopy class of the
loop \(\gamma: t \mapsto e^{2\pi it}\)---here, I am thinking of \(\bS^1\) as
the unit circle centred at the origin embedded in \(\CC\)---to the element \(1
  \in \ZZ\). Intuitively, this makes sense: the circle does geometrically have
a hole and that is what the unit element detects. The fact that the loops
obtained by concatenating \(\gamma\) together give other nontrivial loops is
not all too surprising either: when you wind a string around a ring, you can
certainly tell how many times you wound around. A formal proof of this is a bit
delicate, and perhaps I will return to proving this in the near future. For
now, let's take this on faith, justified by intuitive soundness.

\epoint{Fundamental Group of a Product}
Let \(X\) and \(Y\) be topological spaces and consider the product space
\[ X \times Y \coloneqq \set{(x,y) | x \in X, y \in Y}, \]
and where the topology on \(X \times Y\) has a base given by
\[ \cB \coloneqq \set{U \times V | \text{\(U\) is open in \(X\) and \(V\) is open in \(Y\)}}, \]
i.e. the open sets in the product \(X \times Y\) are unions of products of open
sets from either component. If we define things as such, it appears as if we are
just putting the spaces together---well, something closer to putting a copy of \(X\)
along every point of \(Y\)---and other than that, there is not much interaction
between the two. Thus we might hope that any curve in the product \(X \times Y\)
may be ``separated'' into a product of a curve in \(X\) and then a curve in \(Y\).

Indeed, if we think about it a little, suppose that we have a curve \(\gamma:
  [0,1] \to X \times Y\). We can write out the components of \(\gamma\):
\[ \gamma(s) = (\xi(s),\eta(s)) \]
for curves \(\xi: [0,1] \to X\) and \(\eta: [0,1] \to Y\) going into the
individual spaces. Thus, there is a homotopy from \(\gamma\) to the curve which
first does \((\xi(s),\eta(0))\), and then does \((\xi(1),\eta(s))\); in other
words, the components of \(\gamma\) can be split apart.

What does this mean in terms of the fundamental group \(\pi_1(X \times Y, (x,y))\)?
Roughly speaking, this means that if \(\gamma\) is a loop in \(X \times Y\)
based at \((x,y)\), then we can find another loop \(\xi \times \eta\) in \(X \times Y\)
based at \((x,y)\) which is homotopy equivalent to \(\gamma\), but for which the
movements in \(X\) and \(Y\) are separated. Moreover, it does not matter what sort
of order these movements take place. Hence we may as well think of \(\xi\) as a
loop in \(X\) based at \(x\) and \(\eta\) a loop based at \(y\) in \(Y\) and view
the symbol \(\xi \times \eta\) as a tuple in \(\pi_1(X,x) \times \pi_1(Y,y)\).
Indeed, with a little work, one can make precise the argument here and show
that, as groups, we have
\[ \pi_1(X \times y, (x,y)) \cong \pi_1(X,x) \times \pi_1(Y,y). \]
In words, \emph{the fundamental group of the product is the product of the
  fundamental groups}.

\epoint{A Word about Base Point}
When we defined the fundamental group, we said that the underlying set is the
set of homotopy class of loops at some fixed point. Indeed, we do need this
base point to make concrete sense of the group. However, a few pictures will
convince you that changing base point does not change the group all too much.
Indeed, so as long as the space \(X\) is \emph{path-connected}---every pair of
points in \(X\) has a path connecting them---then the choice of base point is
essentially superfluous.

More precisely, suppose we have two points \(x_1 \in X\) and \(x_2 \in X\)
and let \(\rho: [0,1] \to X\) be a path from \(x_1\) to \(x_2\). Then how do
we go from an element of \(\pi_1(X,x_1)\) to \(\pi_1(X,x_2)\)? Well, an element
in the former is a loop \(\gamma_1\) based at \(x_1\). We can obtain a loop based
at \(x_2\) from \(\gamma_1\) as follows:
\begin{itemize}
  \item We start off at \(x_2\) and travel to \(x_1\) along \(\rho^{-1}\);
  \item Then we take the loop \(\gamma_1\) at \(x_1\);
  \item Then we travel back from \(x_1\) to \(x_2\) along \(\rho\).
\end{itemize}
It is not too hard to show that this map
\[ \pi_1(X,x_1) \to \pi_1(X,x_2), \quad \gamma \mapsto \rho^{-1} \star \gamma \star \rho \]
is an isomorphism of groups. Thus changing the base point for the fundamental
group has the effect of conjugating the group by a path. For that reason,
often, when we are working in path-connected spaces---which is almost always
because I sort of like being sane---we will omit reference to the base point.

\epoint{Boxes, Tori, Cylinders}
Though simple, knowing the three facts:
\begin{enumerate}
  \item \(\pi_1(\RR) = 0\);
  \item \(\pi_1(\bS^1) = \ZZ\); and
  \item \(\pi_1(X \times Y) = \pi_1(X) \times \pi_1(Y)\);
\end{enumerate}
allows us to compute the fundamental group of a large class of objects. A
box is essentially any product of \(\RR\)'s. From (i) and (iii), we have
\[ \pi_1(\RR^n) = 0. \]
A \(n\)-\emph{torus} \(\mathbf{T}^n\) is a product of \(n\) circles: \(\mathbf{T}^n \coloneqq (\bS^1)^n\).
From (ii) and (iii):
\[ \pi_1(\mathbf{T}^n) = \ZZ^n. \]
Finally, a cylinder is some product of boxes and circles. Thus:
\[ \pi_1(\RR^n \times \mathbf{T}^m) = \ZZ^m. \]
\end{document}
