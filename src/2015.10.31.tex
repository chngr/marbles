\documentclass{axolotl}
\title{Topologies}
\author{Raymond Cheng}
\date{October 31, 2015}
\begin{document}
\maketitle
Today, we will start off by recalling some of the basic definitions and results
from topology and differential geometry, in preparation for a study of
algebraic topology and the theory of Lie groups. Let's start off with the
basics and see how far we get.

\epoint{Topological Spaces: Intuition}\label{10-31.1}
Recall that a \textit{topological space} is, formally, a set \(X\) along with a
collection \(\cT\), the \textit{topology} on \(X\), of subsets of \(X\)
satisfying some axioms---here, I used ``collection'' in lieu of ``set'' because
I detest the phrase ``set of subsets'' and derivatives thereof. I will get to
the axioms in a moment. First a discussion. The elements of \(\cT\) are the
\textit{open sets} of the topology. Intuitively, we think of open sets as
telling us which points of \(x\) are ``close together''. The more open sets two
points are in together, the ``closer'' we think they are together; that is, if
two points \(x,y \in X\) are mutually in many open sets \(U \in \cT\) together,
then we think of \(x,y\) as being very close together. This intuition is a bit
limited; more accurately, we should think of an open set \(U \in \cT\) as being
a set containing points of \(X\) such that, for each point \(x \in U\), some
small \textit{neighbourhood}---the term neighbourhood is usually used to refer
to some (open) set containing \(x\)---that are really close by to \(x\) is also
contained in \(U\).

The main example to keep in mind is \(X = \RR^n\)---or any other metric space,
for that matter. Recall that the open sets \(U\) in \(X\) are those for which,
if \(x \in U\), then there is some \(\epsilon > 0\) such that the set of all
points \(B_\epsilon(x)\) that are distance less than \(\epsilon\) away, \[
  B_\epsilon(x) \coloneqq \set{y \in X | \mathrm{dist}(x,y) < \epsilon}, \] is
contained in \(U\): \(B_\epsilon(x) \subseteq U\). In the context of metric
spaces, open sets are really manifesting the intuition stated above.

\epoint{Axioms for a Topology}
Alright, what sort of things should a topology satisfy? I will list them out
without motivation first, and come back to discuss the content of each
immediately afterward. As you are going through the axioms, think back to the
example of a metric space and mentally check that each axiom is satisfied. For
a collection \(\cT\) of subsets of \(X\) to be a topology on \(X\), \(\cT\)
must satisfy the following properties:
\begin{enumerate}
  \item \(\cT\) is closed under \textit{arbitrary} unions: if \(U_\alpha \in
      \cT\), \(\alpha \in A\), with the index set \(A\) of arbitrary cardinality,
    then \(\bigcup_{\alpha \in A} U_\alpha \in \cT\);
  \item \(\cT\) is closed under \textit{finite} intersections: if
    \(U_1,\ldots,U_n \in \cT\), then \(U_1 \cap \cdots \cap U_n \in \cT\);
  \item \(\cT\) contains \(\varnothing\) and \(X\).
\end{enumerate}
Arbitrary unions of opens should be open because when you are making a set
bigger, you do not destroy the property that if you take a point in the open
set, then some set of nearby points is back in the open set. On the other hand,
if you take intersections, you make the set smaller. Any finite process should
not remove very close points, but an infinite process may very well do
so---think of intersecting the open balls \(B_{1/n}(x)\) in a metric space, as
\(n\) ranges over the positive integers. Finally, the last axiom should be
thought of as a nontriviality state, or else one that is almost forced by the
other two for consistency's sake. There is intuitive content in both: since
\(X\) is everything, certainly there is a neighbourhood of every point back in
\(X\); since \(\varnothing\) contains no points, the statements are vacuously
true.

\epoint{Other Notions in Topological Spaces}
Besides open sets, there are other notions that are commonly used in
topological spaces:
\begin{enumerate}
  \item \textit{Closed sets}: a set \(F \subseteq X\) is closed if it is the
    complement of an open set, i.e. there is some open set \(U \subseteq X\)
    such that \(F = X \setminus U\);
  \item \textit{Closure of a set}: Let \(S \subseteq X\) be an arbitrary subset---it need not
    be open nor closed---then the \textit{closure} of \(S\) is the smallest closed
    set \(\overline S \subseteq X\) containing \(S\); this may be constructed by
    taking the intersection over all closed sets containing \(S\);
  \item \textit{Interior of a set}: Let \(S \subseteq X\) be an arbitrary subset, then
    the \textit{interior} of \(S\) is the largest open set \(S^\circ \subseteq X\)
    contained in \(S\); this may be constructed by taking the union of all
    open sets contained in \(S\);
  \item \textit{Boundary of a set}: Let \(S \subseteq X\) be arbitrary, then
    the \textit{boundary} of \(S\) is the intersection of the closure of \(S\)
    with the closure of its complement \(\partial S \coloneqq \overline S \cap
      \overline{(X \setminus S)}\).
\end{enumerate}
These are the main other sets that are useful right now. Think about what each
object is back in a metric space and you will get a good intuitive feel for what
they are.

\epoint{Generating Topologies}\label{10-31.2}
One last thing for today. How are we going to specify a topology on a set
\(X\)? Sure, we can write down a bunch of subsets of \(X\) and declare those
open sets. But then we would have to painstakingly check that the axioms of a
topological space are satisfied. Moreover, there may be some open sets that are
very difficult to specify precisely.

But here's an idea: we can write down some collection of subsets of \(X\) and
then declare that these somehow ``generate'' a topology on \(X\). The intuitive
way of doing this is, well, throw in all those subsets of \(X\) for which should
be in this set for it to be a topology and then declare the expanded set to be
the topology. There are two ways about this.

First, we can specify a collection \(\cB\) of subsets of \(X\) which satisfy
at least the finite axioms. That is, consider a collection \(\cB\) of subsets
of \(X\) which
\begin{enumerate}
  \item is closed under finite intersections;
  \item contains \(\varnothing\) and \(X\).
\end{enumerate}
Then we can define a topology on \(X\) by taking \(\cT\) to consist of the
elements of \(\cB\) along with arbitrary unions of elements of \(\cB\):
\[ \cT \coloneqq \Set{U \subseteq X | U = \bigcup_\alpha U_\alpha, U_\alpha \in \cB}. \]
The set \(\cB\) is called a \textit{base} for the topology \(\cT\).

A second method we can specify a basis is via a subbasis. Consider a collection
\(\sS\) of subsets of \(X\) which we only demand to contain \(\varnothing\) and
\(X\). This is almost like a base, but we are missing the finite intersection closure.
So we proceed as before: we first take all finite intersections of elements in \(\sS\)
to obtain a new collection \(\cB \coloneqq \Set{U \subseteq X | U = U_1 \cap
    \cdots U_n, U_i \in \sS}\). Then we take \(\cB\) as a base of a topology
\(\cT\) on \(X\). In this case, \(\sS\) is said to be a \textit{subbasis} for
the topology \(\cT\).

\epoint{Example: Metric Spaces}
Let's work in a metric space \(X\)---you may well think \(X = \RR^n\) throughout.
A basis for the topology of \(X\) is the collection of open balls
\[ \cB \coloneqq \set{B_\epsilon(x) \subseteq X | x \in X, \epsilon > 0}. \]
For an example of a subbasis, consider the even more specific example of \(X = \RR\).
Then the set
\[ \sS \coloneqq \set{(a,\infty) | a \in \RR} \cup \set{(-\infty,b) | b \in \RR} \]
consisting of open rays is a subbasis for the usual topology on \(X\).
\end{document}
