\documentclass{axolotl}
\title{Topological and Smooth Manifolds}
\author{Raymond Cheng}
\date{2015.11.01}
\begin{document}
\maketitle
Yesterday, we recalled the notion of a topological space. But general
topological spaces can be quite wildly behaved, and your intuition from metric
spaces can lead you far astray. Although this is not a good reason to avoid the
study of general point-set topology, it is at least a deterrent for when we are
not necessarily interested in dealing with pathologies quite yet. We will cross
that bridge when we need to sometime in the future. Rather, let's deal with a
particularly nice class of topological spaces: topological manifolds.

\epoint{Hausdorff Axiom}
Before delving into a discussion of topological manifolds, I should probably
talk about the notion of a Hausdorff topology. The Hausdorff axiom intuitively
says that any two points in the space are separated from one another: think
about two different points in the space, then if they are to be separated, then
there should be some small neighbourhood around each of the points such that
the neighbourhoods are disjoint from one another. If that is not true, then
these points are somehow always very close to one another; they can not be
viewed completely apart from one another.

More formally, let \(X\) be a topological space. We say that the topology on
\(X\) is \textit{Hausdorff} if for every pair of distinct points \(x,y \in X\),
there are open sets \(U,V \subseteq X\) such that \(x \in U\) and \(y \in V\)
and which are disjoint: \(U \cap V = \varnothing\). The Hausdorff condition on
a topology is one example of a \textit{separation axiom}. There are a family of
separation axioms, labelled from T0 to T8 or something---there are some
fractional ones in between---and the Hausdroff one is T2 in that list.

\epoint{Continuous Maps}
We should also talk about what sort of functions we want to consider between
topological spaces. A topology is an abstract way of formulating the notion of
closeness in a space. Thus, a map between topological spaces should be one
which preserves, to some degree at least, the property of two points being
close to one another. Such maps will be called \textit{continuous maps}.

Let \(f: X \to Y\) be a function between two topological spaces. From the
brief discussion before, we need \(f\) to respect the open sets in \(X\) and
\(Y\) in some way. What should this relation relation between open sets be? At
first glance, perhaps we need that \(f\) maps open sets to open sets. That,
however, is not quite what we want: a continuous function should map nearby
points to nearby points, but one thing it can do as well would be map
everything so close to one another that they all become the same. In this case,
nearby points are indeed mapped close by, just too close to tell apart, and
thus not necessarily forming an open set.

On the other hand, if we take a look at points that are near one another in
\(Y\), then for \(f\) to be continuous, they must have come from points in
\(X\) that are close to one another---or simply not have come from \(X\) at
all. In other words, this suggests that the right formulation of continuity
is the following:
\begin{quote}
  \(f\) is \textit{continuous} if for every open set \(V \subseteq Y\), \(f^{-1}(V)
    \subseteq X\) is open in \(X\).
\end{quote}
Wait a minute, here we are defining continuity between topological spaces when
we already have a notion of continuity of functions in metric spaces. Recall
that a function \(f: X \to Y\) between two metric spaces is said to be
continuous if for every \(x \in X\) and every \(\epsilon > 0\), there is some
\(\delta > 0\) such that whenever \(d_X(x,x') < \delta\), then
\(d_Y(f(x),f(x')) < \epsilon\). It is annoying, though not difficult, to check
that these two definitions of continuity coincide. You should do this check.

\epoint{Homeomorphisms}
Now that we have functions between topological spaces, we can ask when a function
should constitute a sort of isomorphism between spaces. As per usual, an isomorphism
of topological spaces---typically called a \textit{homeomorphism}---should be a
continuous map \(f: X \to Y\) which admits an inverse \(f^{-1}: Y \to X\). This
means that a homeomorphism is bijective. Also, we want to the inverse to be a map
of topological spaces, i.e. the inverse is also continuous.

Naturally, we wonder whether or not it is enough to ask for \(f: X \to Y\) to
be continuous and bijective. Well, unfortunately, there are continuous
bijections whose inverse is not continuous. I encourage you to look for an
example. Perhaps looking at rather trivial topologies would give you a simple
example. Otherwise, I think---I am not completely sure right now---there is an
example of such a map between subsets of \(\RR\).

\epoint{Definition of a topological manifold}
Finally, we can come to discuss the notion of a topological manifold! Roughly
speaking, a topological manifold is a topological space that locally looks like
euclidean space \(\RR^n\). More precisely, a topological manifold should be a
topological space \(X\) such that for every \(x \in X\), there is an open set
\(U \subseteq X\) containing \(x\) such that \(U\) is homeomorphic to an open
subset of \(\RR^n\), for some positive integer \(n\). The integer \(n\) is
called the \textit{dimension} of \(X\)---note that we have not shown that \(n\)
is independent of the choice of homeomorphism, but we are going to assume that
it is.

If only the definition were so simple. To ensure that we are working with
sensible spaces, we are going to impose two more technical conditions on the
definition of a topological manifold. First, we are going to demand that \(X\)
is Hausdorff. Second, we are going to ask for \(X\) to be \textit{second
  countable}, that is, that \(X\) posses a countable base \(\cB\) for its
topology. For instance, \(\RR^n\) is second countable because a countable base
for its usual topology is
\[ \cB \coloneqq \set{B_{\frac{1}{k}}(x) | x \in \QQ^n, k \in \ZZ_{> 0}}. \]
Both these technical restrictions are so that topological manifolds are
reasonable spaces, with no strange separation problems and so that they are not
too big. Morally, a topological manifold should be a reasonable topological
space which looks locally like \(\RR^n\).

Let me formulate the ``looks locally like \(\RR^n\)'' in a manner that will
be suited for generalization. For a reasonable topological space \(X\) to be a
topological manifold, we require that there is an open cover \(\sU \coloneqq
  \{U_\alpha\}_\alpha\) of \(X\) such that for each \(\alpha\), there is a
homeomorphism \(\varphi_\alpha: U_\alpha \cong \RR^n\)---since open subsets of
\(\RR^n\) are homeomorphic to the whole space, we may as well take the
homeomorphism to the entire space. A pair \((U_\alpha,\varphi_\alpha)\) is
called a \textit{topological chart} for \(X\) and the set of all charts
\(\{U_\alpha,\varphi_\alpha\}_\alpha\) is called a \textit{topological atlas}
for \(X\).

\epoint{Smooth Manifolds}
Other classes of manifolds differ from a topological manifold in that we
are looking at different model spaces. A topological manifold only demands that
the manifold is topologically the same as \(\RR^n\) locally---in particular,
topological manifolds do not care about the analytic structure on \(\RR^n\). In
contrast, \textit{smooth} manifolds are topological manifolds that do care
about the analytic structure. This additional structure found in a smooth
manifold is encoded in demanding more of the atlas. How might this go?

Well, the difference between a smooth manifold and a topological manifold should
be that we can understand what a smooth map between smooth manifolds are. That
is, we should be able to make sense the statement ``\(f: M \to N\) is a smooth
map'', where \(M\) and \(N\) are smooth manifolds. We can at least make sense
smoothness locally: by precomposing \(f\) with the inverse of a chart
\(\varphi_\alpha^{-1}\) for \(M\) and postcomposing \(f\) with a chart
\(\psi_\beta\) for \(N\), we can view \(f\), locally, as a function between two
euclidean spaces:
\[ \psi_\beta\circ f \circ \varphi^{-1}_\alpha: \RR^m \xrightarrow{\sim} U_\alpha \subseteq M \rightarrow V_\beta \xrightarrow{\sim} \RR^n, \]
where \(V_\beta \subseteq N\) is assumed to contain the image of \(U_\alpha\)
under \(f\). Since this is a map between two euclidean spaces, we can make
sense of smoothness. Then we say that \(f: M \to N\), as a map between
manifolds, is smooth when all the \(\psi_\beta \circ f \circ \varphi_\alpha\)
are smooth.

Okay, what sort of maps should be smooth? A map that certainly should be smooth
is the identity map \(\id: M \to M\). What does the smoothness condition mean
here? Well, consider two charts \((U_\alpha,\varphi_\alpha)\) and
\((U_\beta,\varphi_\beta)\) and suppose that \(U_{\alpha\beta} \coloneqq
  U_\alpha \cap U_\beta \neq \varnothing\). Then when we do the above
construction, we obtain a map
\[ \varphi_{\alpha\beta} \coloneqq \varphi_\beta \circ \id \circ \varphi_\alpha^{-1}: \RR^n \xrightarrow{\sim} U_{\alpha\beta} \xrightarrow{\id} U_{\alpha\beta} \xrightarrow{\sim} \RR^n \]
which is to be smooth. This tells us that, in order to make sense of smooth
maps between topological manifolds, we demand that these \textit{transition
  functions} \(\varphi_{\alpha\beta}\), obtained by composing two charts, must
be smooth functions between euclidean spaces! And indeed, this makes sense. If
we think of charts as putting local coordinates on the manifold, then these
transition functions are changes of coordinates. In order to preserve the
smooth structure of a smooth manifold, all changes of coordinates should be
smooth.

Seeing this, we arrive at the definition of a smooth manifold. A smooth
manifold is a topological manifold \(X\) with a topological atlas
\(\{U_\alpha,\varphi_\alpha\}\) such that each of the transition functions
\(\varphi_{\alpha\beta} \coloneqq \varphi_\beta\circ\varphi_\alpha^{-1}\), when
examined on a domain where they make sense, are smooth functions. In order to
avoid annoying technical issues, we usually say that a smooth manifold consists
of a topological manifold along with a maximal smooth atlas on \(X\); because
this is a formal issue, we will not bother formulating this precisely.

Finally, let us mention when two smooth manifolds are to be considered
isomorphic. Naturally, an isomorphism between smooth manifolds \(M\) and \(N\)
will be a homeomorphism \(f: M \to N\) which is smooth and for which the
inverse \(f^{-1}: N \to M\) is smooth. In that case, we say that \(f\) is a
\textit{diffeomorphism} and that \(M\) and \(N\) are diffeomorphic.
\end{document}
