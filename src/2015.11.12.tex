\documentclass{axolotl}
\title{Maps on Surfaces}
\author{Raymond Cheng}
\date{2015.11.12}
\begin{document}
\maketitle
\epoint{Maps} A \textbf{map} is a celluar embedding of a graph \(G\) into a
surface \(\Sigma\). Here, \textbf{celluar embedding} means that if we take the
complement \(\Sigma - G\) (the image of) \(G\) in \(\Sigma\), then each of the
components are homeormorphic to open discs. In other words, if we view the
vertices of \(G\) in \(\Sigma\) as zero-cells, the edges as one-cells, and the
faces in the graph---this makes sense when we embed \(G\) into a surface---as
two-cells, then this gives a cell decomposition of \(\Sigma\). Two maps are
said to be isomorphic if there is a homeomorphism of the surface which takes
one map to another.

A (now) classical problem is to count the isomorphism classes of maps. To make
everything finite, it is useful to ask the following specific enumerative problem:
for a given surface of genus \(g\), how many (isomorphism classes) of maps are
there with \(f_1\) edges? An alternate question that one can pose is: how many
maps are there with \(f_0\) vertices, \(f_1\) edges, and \(f_2\) faces? Notice
that, by Euler's formula
\[ f_0 - f_1 + f_2 = \chi(\Sigma) = 2g - 2, \]
this date determines the genus of the surface in which the map resides.

The rest of today's marble will not contain much precise mathematics. Rather, I
am going to outline several approaches that have been taking to counting maps
on surfaces, and point out some references containing good introductions and
discussions of the relevant techniques.

\epoint{Enumeration Classically}
As far as I know, the problem of enumerating maps was first formulated and
considered by Tutte, with one of the classic papers being~\cite{Tut}. First,
one does not typically count maps as described above, but rather, one
\emph{roots} the map---this amounts to picking an orientation on an edge,
effectively marking the edge---and count rooted maps. The advantage of rooting
maps is that automorphisms of the maps no longer matter: one of the difficult
things about counting maps is to find a combinatorial construction which does
not produce maps isomorphic to others in the process.

Anyway, the method which Tutte used classically was to analyze what happens when
you contract edges. From a careful analysis of the possible situations that can
arise, he is able to find equations which the generating function for maps (with
certain specifications) must satisfy. Unfortunately, he is only able to give
a solution to the map enumeration problem in a restricted class of cases.

\epoint{Representation Theory of the Symmetric Group}
A more recent method---though the paper~\cite{JV} pioneering this method is now
as old as us!---is to use the character theory of the symmetric group, along
with properties of symmetric functions to find a nice generating series for
maps in arbitrary, though fixed, genus. Roughly speaking, Jackson and Visentin
encode the problem of enumerating maps to a problem of enumerating a certain
triple of permutations in the symmetric group. This enumeration problem depends
only on the cycle type of the permutations involved and thus can be solved by
looking at properties of conjugacy classes in the symmetric group
\(\fS_n\)---recall that the symmetric group \(\fS_n\) is the group for
permutations acting on a set of \(n\) elements. The advantage of being able
to translate the problem in this fashion is that tools from representation
theory, namely character theory of the symmetric group, gives succinct and
elegant formulae for the generating series in question.

This method of enumerating maps with the help of character-theoretic methods
have been further developed to enumerate maps in \emph{locally orientable
  surfaces} by Goulden and Jackson~\cite{GJ}. There, they work with a group
that is slightly more complicated than the symmetric group, but still fairly
easy to understand. Then, with techniques similar to the ones in~\cite{JV},
they are able to obtain a formula for the generating series of maps which has
a similar form.

\epoint{Matrix Integrals}
A fairly curious---and I think rather remarkable---manner in which maps can
be enumerated is via matrix integrals. Think of each entry of a matrix as a
real (or complex) coordinate and then consider integrating these variables over
some space of matrices. For instance, we might right something like
\[ \int_{\cH_n} e^{\tr(M^2)} dM \]
where \(M = (x_{ij})\) is an \(n \times n\) matrix and the domain of integration
is the space of Hermitian matrices:
\[ \cH_n \coloneqq \set{M \in \Mat_{n \times n}\CC | M^\dagger = M^{-1}}, \]
where \(M^\dagger\) is the conjugate transpose of the matrix. These integrals
come up in quantum gravity and other physical problems. It turns out that the
computation of these integrals amount to counting the number of maps with a
certain number of properties---this is essentially the content of Wick's Theorem.

I am not going to develop much more of this theory here, and instead point
you toward two accessible references. The first is a friendly survey~\cite{Zv} by Zvonkin
and the second is a chapter in some book~\cite{Bo}. Finally, a resource that has
more than enough information is the book~\cite{LZ}.

\epoint{Closing}
I have mainly provided references in this Marble. Part of the reason of this is
that I do not recall enough off hand to rail off an article quickly. Another is
that there is plenty to explore and learn in this topic. Moreover, as you
quickly find out with a glance at the table of contents of~\cite{LZ}, there are
many interesting connections in this topic to other subjects. Thus this might
be a nice project to continue to work on and, hopefully, the papers cited in
this Marble will provide you with a starting point for some project.

\begin{thebibliography}{9}
  \bibitem[Bo]{Bo}
    J. Bouttier,
    ``Enumeration of maps''.
    \url{http://arxiv.org/pdf/1104.3003.pdf}.
  \bibitem[GJ]{GJ}
    I. P. Goulden and D. M. Jackson,
    \emph{Maps in Locally Orientable Surfaces and Integrals Over Real Symmetric Matrices},
    Canadian J. Math,

  \bibitem[JV]{JV}
    D. M. Jackson and T. I. Visentin,
    \emph{A character-theoretic approach to embeddings of rooted maps in an orientable surface of given genus.}
    Trans. Amer. Math. Soc.
    \textbf{322} (1990), no. 1, 343--363.

  \bibitem[LZ]{LZ}
    S. K. Lando and A. K. Zvonkin,
    \emph{Graphs on Surfaces and their Applications},
    Encylop\ae dia of Mathematical Sciences \textbf{141},
    Springer-Verlag, 2004, 455pp.

  \bibitem[Tu]{Tut}
    W. T. Tutte,
    \emph{On the enumeration of planar maps},
    Bull. Amer. Math. Soc. 74 (1968),
    no. \textbf{1}, 64--74.
    \textbf{48} (1996), 569--584.

  \bibitem[Zv]{Zv}
    A. K. Zvonkin,
    \emph{Matrix integrals and map enumeration: An accessible introduction},
    Mathematical and Computer Modelling
    \textbf{26}, (1997),
    no. 8–10, 281–-304.
\end{thebibliography}
\end{document}
