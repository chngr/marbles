\documentclass{axolotl}
\title{Euler Characteristic and Genus}
\author{Raymond Cheng}
\date{November 6, 2015}
\begin{document}
\maketitle
\epoint{Euler Characteristic}
The notion of an Euler characteristic is the remarkable observation that a
certain alternating sum associated with a triangulation of a space is actually
an invariant of the space itself. Consider a topological space \(X\) equipped
with a triangulation \(\sT\). Let \(n_i \coloneqq n_i(\sT)\) denote the number
of \(i\)-dimensional simplices in \(\sT\) and assume that the top-dimensional
simplices in \(\sT\) are of dimension \(n\). Then the sum
\[ \chi(X,\sT) \coloneqq \sum^n_{i = 0} (-1)^i n_i \]
is called the \emph{Euler characteristic} of \(X\) with respect to the
triangulation \(\sT\). The remarkable thing about the Euler characteristic is
that it is independent of the triangulation and is an invariant of the space:
that is, if \(\sT\) and \(\sT'\) are two triangulations of a space \(X\), then
\[ \chi(X,\sT) = \chi(X,\sT'). \]
Let me point toward how one might go about proving this statement. First one
shows that if \(\sT'\) is a subdivision of \(\sT\), then the statement is true.
This is not too difficult to show: one simply looks at what happens when you add
an additional top-dimensional simplex and counting the number of faces you add.
Perhaps the only tricky part is the need to add simplices so that it intersects
other simplices at their faces. Once one shows this, observe that given two
triangulations \(\sT\) and \(\sT'\), one can subdivide both \(\sT\) and
\(\sT'\) into a sufficiently triangulation \(\sT''\); that is, given two
arbitrary triangulations, it is possible to find a common subdivision of the
two. Since sudividing a triangulation does not change its Euler characteristic,
it follows that \(\chi\) is independent of the triangulation.

Finally, if \(\varphi: X \to Y\) is a homemorphism and \(\sT\) is a
triangulation of \(X\), notice that \(\varphi(\sT)\) is a triangulation of
\(Y\). Thus \(\chi(X) = \chi(Y)\), which is to say that the Euler
characteristic of a space is homemorphism invariant.

\epoint{Genus}
Drawing a few pictures of triangulations of surfaces, one becomes more and
more convinced that the Euler characteristic is somehow detecting the presence
of holes in the space. Indeed, let's think along these lines for a moment. The
Euler characteristic is an alternating sum of the face numbers of a
triangulation. When there are holes in the space, faces of simplices that might
otherwise be touching are now separated from one another, thus potentially
causing the Euler characteristic to be different. Indeed, it turns out that
in the \(2\)-dimensional case, if \(g\) is the number of holes in the space---for
instance, \(g = 1\) for a torus; \(g = 2\) for a double torus; etc.---then
\[ \chi(X) = 2 - 2g. \]
This gives us a way to define the \emph{topological genus} of higher dimensional
spaces. If \(X\) is a topological space, then its genus \(g(X)\) is defined
\[ g(X) \coloneqq \frac{\chi(X)}{2} - 1. \]

\epoint{Genus Classifying Surfaces}
It turns out that compact, orientable topological surfaces are completely
classified by their genus. Recall that a space \(X\) is said to be
\emph{compact} if for every open cover \(\{U_\alpha\}_{\alpha \in A}\), \(A\)
here is an index set of arbitrary size, there is a finite open subcover, i.e. a
finite subset \(I \subset A\) such that \(\{U_i\}_{i \in i}\) is an open cover
for \(X\). I will skirt over the definition of orientable for the time being,
but just know that the M\"obius band is \emph{not} orientable, whereas the
sphere and the torus are both orientable. Finally, a \emph{surface} is simply
a \(2\)-dimensional topological manifold.

Anyway, compact, orientable surfaces are completely classified by their genus.
Concretely, this means that, essentially, the only such surfaces are the
\(g\)-holed tori \(X_g\), where \(g\) is any nonnegative integer. A cute way of representing
these surfaces is via a \(4g\)-sided polygon. Ask me to show you how this is done,
as it is much more enlightening to do this via a picture. From the diagram, one
can also compute the fundamental group of \(X_g\):
\[ \pi_1(X_g) = \left\langle a_1,b_1,\ldots,a_g,b_g \middle|\;\prod^g_{i = 1} a_ib_ia_i^{-1}b_i^{-1} = 1\right\rangle \]
that is, \(\pi_1(X_g)\) is the group generated by \(2g\) elements \(a_1,\ldots,a_g,b_1,\ldots,b_g\),
subject to the single relation that the product
\[ \prod^g_{i = 1} a_ib_ia_i^{-1}b_i^{-1} = 1. \]
What this means makes a lot more sense with a picture. Do ask me for one!
\end{document}
