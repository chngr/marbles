\documentclass{axolotl}
\title{Triangulations and Cell Decompositions}
\author{Raymond Cheng}
\date{November 5, 2015}
\begin{document}
\maketitle
Today we are going to step back from homotopy a bit and talk about a way of
making a topological space a bit more manageable. Roughly speaking, we are
going to talk about how you might take a topological space \(X\), and then
subdivide it to easier pieces which fit together well. In the nicest case,
these simpler pieces will be triangles---and their higher-dimensional
analogues---and these will be joined together along their faces. Indeed, the
old idea of triangulating a surface is to find a way of putting
triangles in a space so that each of the triangles meet other triangles only
at edges.

\epoint{Simplices}
In order to make sense of triangulations in general topological spaces, we need
to make sense of the analogue of the triangle in higher dimensions. There are
several ways to define this. One simple way is the following. Define the \emph{standard
  \(n\)-dimensional simplex} \(\Delta^n\) to be the set
\[ \Delta^n \coloneqq \set{x \in \RR^n | 0 \leq x_1 + \cdots + x_n \leq 1, x_i \in [0,1]}. \]
A bit of effort, and drawing the low-dimensional examples, should convince you
that this is the right definition of a simplex. Alternatively, \(\Delta^n\) can
be defined as the convex hull
\[ \Delta^n \coloneqq \Conv\set{0,e_1,\ldots,e_n} \]
where \(e_i\) is the \(i\)\ssth~standard basis vector of \(\RR^n\).

Given an arbitrary topological space \(X\), a \emph{\(n\)-simplex in \(X\)} is an
embedding \(f: \Delta^n \to X\)---an \emph{embedding} is an continuous map which
is a homeomorphism on its image---of the standard \(n\)-simplex into \(X\). For
example, an \((n-1)\)-simplex in the \(n\)-simplex is given by the map
\[ f_0: \Delta^{n-1} \to \Delta^n, \quad (x_1,\ldots,x_{n-1}) \mapsto (x_1,\ldots,x_{n-1},0). \]
This is the \((n-1)\)-simplex on the bottom facet of \(\Delta^n\).

In general, the faces of a simplex are important sub-simplices. For later use,
given an \(n\)-simplex \(\Delta^n\), the \(i\)\ssth~face \(\Delta^{n-1}_i\) of
\(\Delta^n\) is the \((n-1)\)-simplex in \(\Delta^n\) given by the map \[ f_i:
  \Delta^{n-1} \to \Delta^n, \qquad (x_1,\ldots,x_{n-1}) \mapsto
  (x_1,\ldots,x_{i-1},0,x_{i},\cdots,x_{n-1}), \] that is, the face in
\(\Delta^n\) for which the \(i\)\ssth~coordinate is \(0\); this definition
makes sense for \(i = 1,\ldots,n\); for \(i = 0\), define \(\Delta^{n-1}_0\) to
be the face given by \(f_0\) above.

Note that the faces discussed here are the maximal dimensional faces of the
simplex. In general, \(k\)-dimensional faces of a simplex \(\Delta^n\) are given
by the set of points for which exactly \(k\) coordinates are nonzero.

\epoint{Triangulations}
A \emph{triangulation} of a topological space \(X\) consists of a set of simplices
\(\sT \coloneqq \{f_\alpha^m: \Delta^m \to X\}\) in \(X\) such that:
\begin{enumerate}
  \item Every point of \(x \in X\) is in some simplex \(f_i^n(\Delta^n)\);
  \item If \(f_\alpha^m: \Delta^m \to X\) is a simplex in \(\sT\), then every
    face \(f_{\alpha,i}^m: \Delta^{m-1} \to \Delta^m \to X\) is also in \(\sT\);
  \item If \(f_\alpha^m: \Delta^m \to X\) and \(f_\beta^n: \Delta^n \to X\) are in
    \(\sT\) such that they intersect, \(f_\alpha^m(\Delta^m) \cap f_\beta^n(\Delta^n) \neq \varnothing\),
    then the intersection is a face of both \(f_\alpha^m\) and \(f_\beta^n\).
\end{enumerate}
A topological space equipped with a triangulation will be called a
\emph{simpicial complex}.

To keep triangulations sane, we typically impose some additional technical restrictions.
One is that triangulations should be \emph{locally finite}: for every \(x \in X\),
there should be an open set \(x \in U \subseteq X\) which intersects only finitely
many of the simplices in the triangulations. If we were being extremely precise, we
would want some condition on how the maps \(f_\alpha\) behave with respect to the
topology on \(X\) and that of the manifold. For the time being, however, we will
go with this fairly simple definition.

\epoint{Simple Examples}
Examples of triangulations should really be drawn out; they are certainly best
understood through pictures. However, let's describe what the formal data of a
triangulation of the interval \([0,1]\) might consist of. Before doing that,
let me remark that \([0,1] = \Delta_1\), i.e. the unit interval is precisely
the standard \(1\)-dimensional simplex. This is immediately seen when you review
the definition of the simplex \(\Delta_1\). This observation gives a simple
triangulation \(\sT\) consisting of:
\begin{enumerate}
  \item A \(1\)-simplex can be mapped in via the identity map: \(f_1^1:
      \Delta_1 \to [0,1]\);
  \item We now need to include the two \(0\)-dimensional faces of \(f_1^1\);
    this means that we need the two \(0\)-simplices:
    \begin{align*}
      f_2^0: & \Delta_0 \to [0,1], \quad \bullet & \mapsto 0, \\
      f_3^0: & \Delta_0 \to [0,1], \quad \bullet & \mapsto 1.
    \end{align*}
\end{enumerate}
It is easy to see that the three simplices here will give a triangulation of
\([0,1]\). Note triangulations are not unique. For instance, we can obtain
another triangulation \(\sT'\) of \([0,1]\) by subdividing the \(1\)-simplex in
\(\sT\) to obtain the pair of simplices
\begin{align*}
  g_1^1: & \Delta_1 \to \left[0,\frac{1}{2}\right], \quad g_2^1: \Delta_1 \to \left[\frac{1}{2},1\right]
\end{align*}
and then include the three \(0\) simplicies mapping to \(0\), \(\frac{1}{2}\)
and \(1\). The triangulation \(\sT'\) is said to be a \textit{subdivision} of
\(\sT\), because every simplex \(g \in \sT'\) is contained in some simplex
\(f\) of \(\sT\)---here, we abuse notation a bit and identify a map \(g:
  \Delta^n \to X\) with its image \(g(\Delta^n) \subset X\).

Notice that if we specify the maximal simplices in the triangulations, then there
is a unique way to complete it into a triangulation: simply include every face
of these simplices. Thus, from now on, we will only specify the maximal
simplices when giving a triangulation.

Let \(X \coloneqq [0,1] \times [0,1]\) be the unit square in \(\RR^2\). The following
can be taken to be maximal simplices for a triangulation of \(X\):
\begin{align*}
  f_1^1: \Delta^2 & \to X             & f_2^1: \Delta^2 & \to X \\
         (x_1,x_2) & \mapsto (x_1,x_2), & (x_1,x_2) & \mapsto (1 - x_1,1 - x_2).
\end{align*}
If you draw a picture of what these two mean, the simplex \(f_1^1\) is the
triangle in \(X\) with vertices \((0,0)\), \((1,0)\) and \((0,1)\), while the
simplex \(f_2^1\) is the triangle with vertices \((1,0)\), \((0,1)\) and
\((1,1)\). These two simplices induce a valid triangulation of \(X\) because
they intersect along the line between \((1,0)\) and \((0,1)\), which is a face
of both simplices.

For an example for something that is not a triangulation, you can consider the
``triangulation'' obtained by subdividing \(f_2^1\) in half by cutting it from
\((1,1)\) to the centre of the square to obtain two smaller triangles \(g_1\)
and \(g_2\). This will not be a triangulation---despite \(X\) being covered by
triangles!---because the intersection between \(g_1\) and \(f_1\), say, is now
no longer a face of \(f_1\).
\end{document}
