\documentclass{axolotl}
\title{Homotopy}
\author{Raymond Cheng}
\date{November 2, 2015}
\begin{document}
\maketitle
Yesterday, we talked about topological manifolds and also mentioned smooth
manifolds at the end. Before talking more about smooth manifolds, on which
analytic techniques give a rich theory, let look a little into the theory of
general topological spaces, with an emphasis on understanding topological
manifolds---I sort of pretend that the only topological spaces out there are
topological manifolds and hide my head in a hole when pathological topologies
spring about.

We shall discuss a fundamental notion in topology, that of homotopy. The idea
simple: imagine a curve in space and consider deforming the curve in a
continuous fashion so that the end points remain fixed. This is the geometric
picture you should have in your head when thinking about homotopy. But let
us make this more precise.

\epoint{Homotopy of Curves}
Okay, so what is homotopy supposed to be? It needs to formalize the idea of
deforming a space in a continuous manner in a topological space. Okay, well let
us consider how we might formalize this for a curve \(\gamma\) in a topological
space \(X\). Parameterizing \(\gamma\) by the unit interval \(I \coloneqq [0,1]\), we can think of
the curve as a continuous map \(\gamma: I \to X\). Now suppose that we can
deform \(\gamma\) into another curve \(\gamma'\) in \(X\) in some continuous
manner. Well, for that to happen, you should be able to show me a way to bend
\(\gamma\) continuously to obtain \(\gamma'\); put differently, we should able
to film a time-lapse such that at time \(t = 0\), we see an image of \(\gamma\)
and as \(t\) tends to \(1\), the image changes continuously so that at \(t =
  1\) we see \(\gamma'\).

This suggests that we define a \emph{homotopy from \(\gamma\) to \(\gamma'\)}
as a function
\[ H(s,t): [0,1] \times I \to X \]
which is continuous in its argument \(t\)---think of time as continuously
moving along---and for each fixed time \(t\), the function
\[ H_t(s) \coloneqq H(t,s): I \to X \]
is a continuous function in the argument \(s\)---at each time, the curve that
we see should be continuous. Also, to be a homotopy \textit{from} \(\gamma\)
\textit{to} \(\gamma'\), we would want that
\[ H_0(s) = \gamma \qquad\text{and}\qquad H_1(s) = \gamma' \]
as continuous functions from \(I\) to \(X\).

\epoint{Homotopy More Generally}
Notice that there was nothing special about the domain \(I\) used in the
definition of homotopy: we can very well consider homotopies between continuous
functions \(\gamma: I \to X\) and \(\gamma': I \to X\) where \(I\) is now an
arbitrary topological space. Just so that the notation is not too misleading
later on, let us switch from using \(\gamma\) and \(\gamma'\) to the more general
\(f\) and \(g\) in referencing continuous functions from \(I\) to \(X\).

With this notation, a \textit{homotopy} between continuous functions \(f: I \to X\)
and \(g: I \to X\) is a map
\[ H: [0,1] \times I \to X \]
such that the maps
\[ H_t(s) \coloneqq H(t,s): I \to X \]
are continuous functions for each fixed \(t \in [0,1]\) and that \(H\) starts at
\(f\) and ends at \(g\),
\[ H_0(s) = f(s) \qquad\text{and}\qquad H_1(s) = g(s). \]

\epoint{Main Example of Homotopy}
Homotopy amounts to deforming a beginning map \(f: I \to X\) to an ending map
\(g: I \to X\) in some continuous fashion within the space \(X\). If \(X\) is
nice enough, there is one easy way in which we can change \(f\) to \(g\)
continuously: we simply linearly interpolate all the intermediate values. More
precisely, for each \(s \in I\), think of drawing a line \(\ell_s\) in \(X\) from the point
\(f(s)\) to \(g(s)\). We can then parameterize \(\ell_s\) by the unit interval
\([0,1]\) and define a homotopy by traversing this line at a constant speed to
get from \(f\) at \(t = 0\) to \(g\) at \(t = 1\). I would draw a picture here,
but that is difficult in \LaTeX---ask me to do when you see me! More formally,
we can define a homotopy \(H: [0,1] \times I \to X\) by
\[ H(t,s) \coloneqq (1 - t)\,f(s) + t\, g(s). \]
Let's check that this is indeed a homotopy from \(f\) to \(g\):
\begin{enumerate}
  \item For each fixed \(t \in [0,1]\), \(H_t(s) = (1 - t)\,f(s) + t\, g(s)\) is
    the sum of continuous functions and thus is continuous;
  \item At \(t = 0\), \(H_0(s) = f(s)\);
  \item At \(t = 1\), \(H_1(s) = g(s)\).
\end{enumerate}

\epoint{Everything is Homotopic in \(\RR^n\)}
As an application of the example above, we can show that every continuous map
\(f: I \to \RR^n\) is homotopic to every other continuous map. It suffices to
show that \(f\) is homotopic to the constant map \(\hat 0: I \to \RR^n\), where
\[ \hat 0(s) \coloneqq 0 \in \RR^n \qquad\text{for every \(s \in I\)}. \]
Indeed, this is because ``being homotopic to'' is an equivalence relation,
something that we shall show later.

Anyway, given a continuous function \(f: I \to X\), a homotopy to the constant
map is constructed by directly using the construction above: consider \(H:
  [0,1] \times I \to X\) given by
\[ H(t,s) \coloneqq (1-t)\,f(s). \]
This is a homotopy which begins at \(f\) and ends at \(\hat 0\).

\epoint{Homotopy is not Trivial in \(\RR^2 - \{0\}\)}
The previous example suggests that homotopy is sort of trivial: we can mash
everything into a constant map in \(\RR^n\). But this is because there is so
much open space with no strange punctures in \(\RR^n\). This is not the case in
the punctured real plane \(X \coloneqq \RR^2 - \{0\}\). If we think about it for
a moment, if we start off with a circle around the origin, then we cannot deform
the circle to a point \textit{while remaining completely inside \(X\)} and
\textit{without cutting the circle}. As a physical picture, think of the
puncture in the plane as coming from a rod from some higher dimension ramming
through the origin. Then think of a circle around the origin as a rubber band
that is encircling the rod. Then there is no way to squish the rubber band so
that it gets squished into a ball with nothing between its ends. This expresses
the fact that a circle is not homotopic to a point in \(X\).

Let's prove that the circle is not homotopic to a point. Suppose for sake of
contradiction that the circle \(\bS^1\) in \(X\) were homotopic to a point \(p
  \in X\). Without loss of generality, assume \(\bS^1 = \set{(x,y) \in X | x^2
    + y^2 = 1}\) and \(p = (1,0)\). If there were a homotopy from \(\bS^1\) to
\(p\), we can consider what happens to the subset \(R\) of \(\bS^1\) that is
left of the \(y\)-axis. Since \(p\) is right of the \(y\)-axis and homotopies
are continuous, the Intermediate Value Theorem implies that, each point of
\(R\) must cross the \(y\)-axis at some time and point. Since there are points
of \(R\) that are both above and below the \(x\)-axis, there will be some
points which cross the \(y\)-axis from above and below the \(x\)-axis. But by
continuity, this means that some point must go through the origin during this
homotopy. But this is impossible! There is no origin in \(X\)! Thus no homotopy
between \(\bS^1\) and \(p\) can exist.
\end{document}
