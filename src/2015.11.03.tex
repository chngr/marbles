\documentclass{axolotl}
\title{Homotopy as Equivalence Relation and \(\pi_1\)}
\author{Raymond Cheng}
\date{November 3, 2015}
\begin{document}
\maketitle

\epoint{Homotopy is an Equivalence Relation}
The statement in the title is so fundamental to much of topology that I am going
to repeat it as the section title here. The intuitively clear statement that
homotopy does define an equivalence class is indeed true, and a little effort
will be required to show this is the case formally. This exercise hopefully
enlightens how homotopy works, so I will be reasonably thorough with this.
Recall that an equivalence relation on a set \(X\) is a relation \(\sim\) such
that
\begin{itemize}
  \item \textbf{Reflexivity.} For all \(x \in X\), \(x \sim x\);
  \item \textbf{Symmetry.} If \(x \sim y\), then \(y \sim x\) for all \(x,y \in X\);
  \item \textbf{Transitivity.} If \(x \sim y\) and \(y \sim z\), then \(x \sim z\), for all
    \(x,y,z \in X\).
\end{itemize}
Let's show that the relation \(\sim\) such that, for continuous maps \(f,g: I \to X\),
\begin{equation*}
  f \sim g \iff \text{\(f\) is homotopic to \(g\)}
\end{equation*}
is an equivalence relation. This is fairly simple:
\begin{itemize}
  \item For \(f: I \to X\), then a homotopy from \(f\) to \(f\) is the constant
    homotopy: \(H: I \times [0,1] \to X\) defined to be \(H_t = f\) for all
    \(t \in [0,1]\);
  \item For \(f: I \to X\) and \(g: I \to X\) homotopic, that is, there is a
    homotopy \(H\) such that \(H_0 = f\) and \(H_1 = g\); now define \(H': I \times [0,1] \to X\)
    by
    \[ H'(s,t) \coloneqq H(s,1-t), \]
    i.e. we reverse the homotopy in time;
  \item For \(f, g, h: I \to X\) three continuous maps with a homotopy \(H: I \times [0,1] \to X\)
    between \(f\) and \(g\), and a homotopy \(H': I \times [0,1] \to X\), then
    a homotopy \(H'': I \times [0,1] \to X\) between \(f\) and \(h\) is obtained
    by first running \(H\) at twice the speed, then running \(H'\) at twice the speed
    right after:
    \[ H''(s,t) \coloneqq \begin{dcases}
        H(s,2t) & t \in \left[0,\frac{1}{2}\right], \\
        H'\left(s,\frac{2t - 1}{2}\right) & t \in \left[\frac{1}{2},1\right].
      \end{dcases}
    \]
\end{itemize}
To make sense of the above, draw a picture! Or ask me to draw a picture!
Hopefully this is all pretty intuitive, though!

\epoint{Fundamental Group: Intuitively}
The reason why we will want to make sure that homotopy is an equivalence
relation is so that we can speak of homotopy equivalence classes. This is
needed to construct the fundamental group of a space. Roughly speaking, the
fundamental group \(\pi_1(X)\) of a space \(X\) looks at all the loops in the
space---to be precise, we will want to look at loops based at some point \(x_0
  \in X\), but we will burn this bridge when we cross it. The aim in looking
at all the loops is that this should somehow measure whether or not there are
``holes'' in the space. Indeed, recall from yesterday that \(\RR^n\) is contractible.
In particular, we say that all loops, say based at the origin, may be homotoped
to a single point. Thus the homotopy classes of loops in \(\RR^n\) is trivial,
and has a single class with representative the constant loop.

On the other hand, we saw that the circle is \emph{not} homotopic to a point in
the punctured plane \(X \coloneqq \RR^2 - \{0\}\). This means that there is
a nontrivial loop in \(X\), so the homotopy classes of loops in \(X\) has more
than the class of the constant loop. We think of the non-triviality of this set
as indicating the presence of a ``hole'' in the space; it is detecting the fact
that we took out the origin in \(\RR^2\).

\epoint{Fundamental Group: Definition}
Let us define the \emph{fundamental group} \(\pi_1(X,x)\) of a space \(X\) with
base point \(x \in X\). First define
\[ \Omega(X,x) \coloneqq \set{\gamma: [0,1] \to X | \text{\(\gamma\) is continuous and \(\gamma(0) = x\), \(\gamma(1) = x\)}} \]
as the set of loops in \(X\) based at \(x\). Then, the set underlying
\(\pi_1(X,x)\) is the set of equivalence classes \(\Omega(X,x)/\sim\) of loops
in \(\Omega(X,x)\), with two loops being equivalent if and only if they are
homotopic.

We say that \(\pi_1(X,x)\) is a group, suggesting that we have some sort of
multiplication on \(\pi_1\). Intuitively, the multiplication is simply concatenation
of paths. More formally, for \(f,g: [0,1] \to X\) loops representing equivalence
classes in \(\pi_1(X,x)\), define the product \(f \star g: [0,1] \to X\) as the
path
\[ (f \star g)(t) \coloneqq \begin{dcases}
    f(2t) & t \in \left[0,\frac{1}{2}\right], \\
    g(2t - 1) & t \in \left[\frac{1}{2},1\right],
  \end{dcases}
\]
which is that we go along \(f\) at twice the speed, then we go along \(g\) at
twice the speed.

We should check that this product actually satisfies the group axioms. This is
also a good time to remind you what those axioms are.
\begin{itemize}
  \item \textbf{Identity.} There is an identity for this product, given by the
    constant path \(x: [0,1] \to X\), \(x(t) \coloneqq x\) for all \(t\). If
    \(f: [0,1] \to X\) is a loop in \(\Omega(X,x)\), then a homotopy \(H: [0,1]
      \times [0,1] \to X\) between \(x \star f: [0,1] \to X\) is
    \[ H(s,t) \coloneqq (x \star f)\left(\frac{t}{2} + \frac{(2 - t)s}{2}\right). \]
    The idea of this homotopy is that at \(t = 1\), we need to start the loop after
    the half way point, that is, after the transition from \(x\) to \(f\). The
    fact that \(x\) is a right identity also comes from a similar homotopy.
  \item \textbf{Inverses.} Inverses for the product may be obtained by concatenating
    a loop \(f: [0,1] \to X\) with its reverse loop, \(\tilde f: [0,1] \to X\),
    \[ \tilde f(t) \coloneqq f(1 -t), \]
    obtained from \(t\) by running time backwards. To see that the product
    \(f \star \tilde f\) is homotopic to the identity, consider a homotopy
    defined as
    \[ H(s,t) \coloneqq \begin{dcases}
        x & s \in \left[0,\frac{t}{2}\right], \\
        (f \star \tilde f)(2s - t) & s \in \left[\frac{t}{2}, 1 - \frac{t}{2}\right], \\
        x & s \in \left[1 - \frac{t}{2},1\right].
      \end{dcases}
    \]
    Ask me to draw a picture for this one. I would not get it otherwise either.
  \item \textbf{Associativity.} There is one last thing we need to check. If
    we have three loops \(f,g,h: [0,1] \to X\), then we need
    \[ (f \star g) \star h = f \star (g \star h). \]
    This shouldn't be too hard to see how this is done, but you can ask me for
    a picture for this one too.
\end{itemize}
\end{document}
