\documentclass{axolotl}
\title{Cell Decompositions}
\author{Raymond Cheng}
\date{November 9, 2015}
\begin{document}
\maketitle
\epoint{Cells}
I have spoken about triangulations of a topological space. The main idea in
such a construction is that simplices are nice, and if we are able to divide
the space up into a collection of simplices that fit together
nicely---simplices only meet one another at a face---then we have a better
chance of understanding the topology of the space. For instance, we found that
triangulations detect whether or not the space has any holes---that is, from a
triangulation of the space, we can compute the topological genus.

However, triangulations are very special. First of all, each simplex has only
finitely many faces, so there are only finitely many other simplices that can
intersect any simplex in a triangulation. Moreover, the intersection condition
that simplices intersect each other in faces is rather stringent as well.

A more general sort of complex---that is, manner in which we might decompose
a topological space into simpler objects---is then perhaps desirable. One of
the most flexible sort of this structure is that of a \emph{CW-complex}. You
can think of a CW-complex as being like a triangulation, but now we are allowed
to use any space that is homeomorphic to a ball. Moreover, the intersection
condition is much more relaxed as we no longer need things to fit together
along a face of a simplex.

Before going on to define a CW complex, let us refer to a \emph{closed
\(n\)-cell} in a topological space \(X\) as the image of a closed ball
\(f: \overline B^n \to X\) under some embedding \(f\). Since an \(n\)-simplex
is homeomorphic to a closed ball, simplices in \(X\) are an example of a closed
cell in \(X\). Likewise, we refer to the image of an open ball \(f: B^n \to X\)
under an embedding as an \emph{open \(n\)-cell} in \(X\).

\epoint{CW Complexes}
A \emph{CW complex} is a Hausdorff topological space \(X\) along with a
partition of \(X\) into open cells \(\varphi_\alpha^{n_\alpha}: B^{n_\alpha}
  \to X\), where the dimensions \(n_\alpha\) may vary from cell to cell. Here,
by partition, we mean that each cell is disjoint from every other and that every
point of \(X\) is contained in some cell. We require these cells to fit together
nicely in the follow two ways:
\begin{enumerate}
  \item For each \(n\)-dimensional cell \(\varphi_\alpha^n: B^n \to X\), there
    exists a continuous map \(\overline\varphi: \overline B^n \to X\) such that:
    \begin{enumerate}
      \item The restriction of \(\overline\varphi\) to the interior \(B^n
          \subset \overline B^n\) is a homeomorphism onto
        \(\varphi_\alpha^n(B^n)\); and
      \item The image of \(\partial B^n\) under \(\overline\varphi\) is contained
        in the union of finitely many cells \(\varphi^{n_\beta}_\beta\), with
        \(n_\beta < n\);
    \end{enumerate}
  \item A subset of \(X\) is closed if and only if it meets the closure of each
    cell in a closed set.
\end{enumerate}
The first condition is called \emph{Closure finiteness} and the second is
\emph{Weak topology}, hence the name CW complex. (Entertaining aside: CW
complexes were introduced by J.H.\textbf{C}. \textbf{W}hitehead. At their time
of introduction, some people accused Whitehead of naming these structures after
himself.)

\epoint{Explanation of the Conditions}
The first condition in the definition of a CW complex is an analogue of the
condition that simplices meet in faces for a triangulation. What this condition
is saying is that the ``open cells'' in the data of a CW complex should be
thought of as closed cells in the space. These closed cells fill up the whole
space and only the boundary of cells intersect with other cells of the same
dimension. When two cells intersect, there are (finitely many) lower
dimensional cells that decompose the intersection.

The second condition is a technical condition saying that the topology of the
space \(X\) behaves well with the topology of the cells. Really, all this is
saying is that we can reasonably understand what is closed in \(X\) just by
looking at the restriction to each cell.

\epoint{Examples}
Let's look at some simple examples of cell-decompositions.
\begin{enumerate}
  \item Any triangulation is a CW decomposition.
  \item If you take the closed ball \(\overline B^2\) in the plane and you identify all the points
    to a single point---say the North pole---then you obtain the sphere
    \(\bS^2\). With this, it is easy to see that a cell decomposition of
    \(\bS^2\) is given by two cells: one is the two-dimensional cell \(B^2\)
    mapping to everything but the North pole, i.e. to the interior of the open
    ball from the previous construction; and there is a zero-dimensional cell
    mapping to the North pole.
  \item Think of a finite graph \(G\) as a topological space in \(\RR^3\),
    where the edges do not intersect. A cell decomposition of \(G\) is given by
    mapping a one-cell to each edge so that the boundary of each one-cell is on
    the vertices, and then mapping a zero-cell to each vertex.
  \item As an extension of the previous example, consider a graph \(G\) and
    suppose you embed it into some surface \(S\)---that is, find a surface
    \(S\) on which you can draw \(G\) so that no edges intersect one another.
    As a simple example, \(S\) can be the sphere and then \(G\) can be a
    drawing of any planar graph in \(S\). We say that \(G\) is
    \emph{celluarly-embedded} if the connected components of the complement \(S
      - G\) is a disjoint union of open discs---that is, the faces are open
    discs.
\end{enumerate}
\end{document}
